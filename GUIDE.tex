\pdfgentounicode=0
\documentclass[nopdfbookmarks,a4paper, 11pt, twoside]{article}

\usepackage{dsekcommon}
\usepackage{dsekdokument}
\usepackage{tabularx}
\usepackage[T1]{fontenc}
\usepackage[utf8]{inputenc}
\usepackage[swedish]{babel}
\usepackage{url}
\usepackage[dvipsnames]{xcolor}

\setheader{GUIDE}{Guide till \LaTeX }{\today}
\title{Guide till \LaTeX }
\author{Kaspian Jakobsson}
\definecolor{datarosa}{HTML}{F280A1}

\begin{document}
\section*{Guide till \LaTeX }
\subsection{Introduktion}
Detta är ett overleaf projekt som har \LaTeX-mallarna i en mapp.
{\verb|./tex/|} innehåller alla stilfiler, medan alla dokumentmallar finns i {\verb|./mallar/|} . 
Klicka uppe i vänstra hörnet, på {\verb |Menu|}. Där kan du välja \verb|Copy project|, vilket kopierar hela projektet till ditt konto. 
\subsection{Hur gör jag ett dokument?}
För att skapa ett nytt dokument kopierar du innehållet i den mall du vill använda, till exempel {\verb|dsekmotion.tex|}, och sedan kopierar du in det i en ny fil som du döper till något, {\verb|dokument.tex|}. Därefter fyller du i alla kommandon som är deklarerade i headern och förhoppningsvis har du en vacker, egen ny pdf-fil där du har fått med dig allt du velat uttrycka. 

\subsection{Hur gör jag en ny mall?}
Kika på de andra mallarna och försök pussla ihop något som är funktionellt! \Dsymbol[3mm]-latexmallarna sköter all stil, och man kan med fördel dyka ned i stilmallarna för att förstå hur allt fungerar.

\subsection{Hur fungerar latexmkrc?}
Den innehåller alla sökvägar som en tex-distribution normalt sett skulle leta efter, t.ex. texinputs, texfontmaps etc. Det är den som gör så att vi kan använda alla DSek-mallar i en annan mapp, och dessutom kunna använda \verb|\lthsymb|, det vill säga:


\textcolor{datarosa}{\Huge \tlth  \fsek \esek \msek \vsek \asek \ksek \dsek \wsek} 

\subsection{Vidare läsning}
\renewcommand\UrlFont{\color{datarosa}\rmfamily\small}
\def\arraystretch{1.5}
\begin{tabularx}{1.0\textwidth} { XlX }
   \url{https://github.com/Dsek-LTH/stadgar}&---& D-Sektionens githubrepo för stadgarna  \\
   \url{https://github.com/Dsek-LTH/reglemente}&---& D-Sektionens githubrepo för reglementet  \\
   \url{https://www.dsek.se/aktiva/grafiskprofil/latex.php}&---& Den gamla \LaTeX-mallen, men sigillet är inte uppdaterat. Finns massvis med guider dock kring de olika dokumenten! \\
\url{https://www.dsek.se/aktiva/grafiskprofil/latex/dsek-latex.zip}&---& Direktlänk till \LaTeX-filerna.  \\
\url{https://www.dsek.se/aktiva/motion/dsek-tex.zip}&---& URL till \TeX-mallen som bl.a. innehåller motionmallen.  \\
\url{https://www.dsek.se/arkiv/dokument.php}&---& Dokument och mötesprotokoll! Titta här för inspiration, finns massvis bland bilagorna \\
\url{https://github.com/Dsek-LTH/dsek-latex}&---& D-Sektionens GitHub-repo för alla klasser och stilmallar. 

\end{tabularx}

\end{document}