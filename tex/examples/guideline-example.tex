\documentclass{dsekguideline}

\setadoptedon{VTM Extra 2021}
\setrevisedon{VTM Extra 2023}

\riktlinjeför{överlämning}

\begin{document}
\maketitle
\section{Formalia}
\begin{parasection}

\paraitem{Sammanfattning}
Denna riktlinje beskriver hur överlämningar på D-sektionen ska gå till.

\paraitem{Syfte}
Syftet med denna riktlinje är att ge funktionärer rättigheter och säkerhet kring
att överlämning sker och vad den i grund och botten ska innehålla.

\paraitem{Historik}
Utkast färdigställt av
\begin{itemize}
  \item Joel Bäcker, Skattmästare 2021,
  \item Josefin Wetterstrand, Vice Ordförande 2021,
  \item David Jobrant, Informationsansvarig 2021, och
  \item Victor Winkelmann, Näringslivsansvarig 2021.
  \end{itemize}
Antagen VTM Extra 2021.

Dokumentent omformades från att vara en policy till att bli en riktlinje på
VTM Extra 2023.

\end{parasection}

\section{Bakgrund}
\begin{parasection}
  \paraitem{Bakgrund}
  Överlämning bland funktionärer har varierat i kvalitet under de gångna
  åren. Denna riktlinje ämnar därför garantera att varje överlämning innehåller
  särskilda moment som säkerställer att den nyinvalda har tillräcklig kunskap
  och är införstådd om vad sektionsarbetet innebär.
\end{parasection}

\section{Överlämning}
\begin{parasection}

\paraitem{Moment i överlämning}
Följande moment ska gås igenom under överlämningen med den tillträdande
funktionären:
\begin{itemize}
    \item vad som åligger posten enligt styrdokumenten,
    \item postens testamente,
    \item ekonomiska rutiner,
    \item rundvandring av de lokaler som posten använder sig av, samt vilken
      access posten ger tillgång till. Eventuella regler för access och
      lokalerna som berörs bör också genomgås,
    \item Vilka förmåner som ges med posten (kan exempelvis vara tack, tröjor
      och medaljer) och
    \item Kunskapsbank för posten (kan exempelvis vara inventeringslistor och
      andra dokument från det gångna året).
\end{itemize}
\paraitem{Tid}
Momenten i överlämningen ska ha genomförts innan personen tillträder sin
funktionärspost. I händelse av att personen har fyllnadsvalts kan man frångå
detta, men överlämningen ska ske snarast möjligen.

\paraitem{Ansvarig}
Ansvarig för att överlämningen genomförs är i första hand nuvarande innehavaren
av posten. Då denna post är vakant åläggs det styrelsen att detta sker. Det är
upp till styrelsen att från fall till fall bedöma om de själva kan arrangera
överlämningen eller om man kan be någon annan om hjälp, exempelvis äldre
företrädare.

\end{parasection}

\section{Testamente}
\begin{parasection}
  \paraitem{Definition} Testamente är ett dokument som ska fungera som stöd för
  en funktionär som är ny på en post. Det är en sammanfattning av postens
  arbetsuppgifter, tips på hur dessa kan genomföras och erfarenheter från
  tidigare år. Testamentet bör uppdateras vid slutet av varje mandatperiod med
  syftet att föras vidare till efterträdaren.

  Inom vissa utskott kan det finnas ett övergripande testamente som täcker hela
  mästeriets verksamhet. I dessa fall kan de två komplettera varandra så alla
  punkter nedan täcks.

  \paraitem{Innehåll}
  Följande delar bör ingå i ett testamente:
  \begin{itemize}
    \item vad som ska göras innan mandatperiodens början,
    \item postens ansvarsområden och återkommande uppgifter,
    \item oavslutade arbetsuppgifter och projekt som lämnas över,
    \item tidslinje, i den mån det är relevant,
    \item hemsidor och dylikt som man ska ha tillgång till, samt
    \item kontaktuppgifter till externa parter.
  \end{itemize}

\paraitem{Arkivering}
Testamentet bör förvaras åtkomligt så framtida efterträdare och mästerier kan ta
del av det.

\end{parasection}
\end{document}
